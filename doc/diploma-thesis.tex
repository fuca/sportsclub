%\documentclass[11pt,oneside]{/home/fuca/Projects/mu-diploma-thesis/doc/fithesis-src/fithesis2c-master/fithesis/fithesis2}
\documentclass[11pt,oneside]{fithesis}
\usepackage[plainpages=true, pdfpagelabels, unicode]{hyperref}
\usepackage{czech}
\usepackage[utf8x]{inputenc}
\usepackage{verbatim}
\usepackage[pdftex]{graphicx}
\usepackage{pdfpages}
%%% \DeclareGraphicsExtensions{.jpg,.pdf,.tif,.png,.tiff}
\usepackage{makeidx}
\makeindex
\thesistitle{Informační systém pro sportovní kluby}
\thesissubtitle{Diplomová práce}
\thesisstudent{Michal Fučík}
\thesiswoman{false}
\thesisfaculty{fi}
\thesisyear{podzim 2014}
\thesisadvisor{RNDr. Michal Batko, Ph.D.}

\hypersetup{
  colorlinks=false,
  pdfborder={0 0 0}
}

\begin{document}
\FrontMatter
\ThesisTitlePage
\begin{ThesisDeclaration}
\DeclarationText
\AdvisorName
\end{ThesisDeclaration}

\begin{ThesisThanks}
Rád bych poděkoval vedoucímu práce RNDr. Michalu Batkovi, Ph.D. za cenné rady a připomínky při tvorbě této diplomové práce. Další poděkování bych chtěl směrovat k rodině a přátelům za poskytnutou podporu.
\end{ThesisThanks}

\begin{ThesisAbstract}
Tato diplomová práce se zabývá tématem informačního systému pro sportovní kluby v podobě samostatné aplikace. Důraz je kladen na nezávislost a konfigurovatelnost jednotlivých komponent systému. Návrh je tvořen na základě průzkumu veřejných potřeb. Datová základna je realizována za pomoci relačího databázového systému. Aplikační vrstva je implementována za pomoci inovativního frameworku Nette, využívajícího jazyka PHP5.

\end{ThesisAbstract}

\begin{ThesisKeyWords}
informační systém, sportovní klub, modularita, webdesign, responzivita, html, php, ajax, css, doctrine, nette
\end{ThesisKeyWords}

\MainMatter
\tableofcontents















%%%%%%%%%%%%%%%%%%%%%%%%% ZADANI %%%%%%%%%%%%%%%%%%%%%%%%%%
\chapter{Zadání diplomové práce}
Zde bude vložen list podepsaného zadání obdrženého na studijním oddělení


















%%%%%%%%%%%%%%%%%%%%%%%%% UVOD %%%%%%%%%%%%%%%%%%%%%%%%%%
\chapter{Úvod}
\paragraph*{}
S neustále zrychlující se a rozvíjející se dobou je čím dál tím větší potřeba informačních technologií v nejrůznějších odvětvích každodenního života. Oblast sportu není výjimkou. S rostoucím zájmem o sport logicky roste i členská základna a tím i množství úsilí potřebné pro věnování ke správě těchto dat. Rozvíjející se kluby se s rostoucí profesionalitou dostavávají k většímu množství prostředků.
Na druhé straně tu však existuje i amatérská sportovní scéna, kde jsou a stále vznikají menší zájmové skupiny, či sportovní kluby, které svou úrovní nedosáhnou na tak tolik prostředků, jako jejich větší \textit{kolegové}. Tyto sportovní organizace jsou z naprosté většiny závislé na dotacích od externích sponzorů a nemohou si nákladné (\cite{analyza-sis}) implementace informačních systémů na míru dovolit. Jsou proto odkázáni na ne vždy vyhovující, či levné řešení. 


- TODO vliv informacniho systemu na proces managementu sportovniho klubu (ma cenu to sem psat, kdyz by to byly stejne jen dve vety, ktery jsou podobne formulovany dal v analyze???)
        \section{Motivace}

        I přes to, že se několik prací na toto téma už objevilo, většina jich pojednává spíše o analýze managementu systému. Když už dojde na implementaci samotného systému, tak je výsledkem většinou aplikace zaměřená na potřeby konkrétního klubu. Potřeby širší veřejnosti však zůstávájí neuspokojeny.

        \section{Cíle práce}

        Zmapovat aktuální stav a rámcové požadavky amatérské sportovní scény z hlediska využití a potřeby existence informačního systému pro lepší správu a management vnitřních politik sportovních klubů.
        \paragraph*{}
        Co se implementace práce týče, je vhodné, aby byly jednotlivé části srozumitelné a jednoduše upravitelné dle rozvíjejících se potřeb. Implementační detaily a informace poskytnuté v dalších částech této diplomové práce by měly sloužit jako popis a návod k rozšíření aplikace o potřebné moduly.

        \section{Struktura práce}

        V úvodní části práce je čtenář seznámen s aktuální dostupností informačních systémů nízkorozpočtových amatérských sportovních klubů. 

        V následující části je uvedena rešerše stávajících řešení na trhu a jejich aplikovatelnost. 

        Další část práce poskytuje náhled na analýzu informačních systémů s aplikací na vyvíjený systém. Spolu se sběrem požadavků od koncových uživatelů v podobě vyhodnocení dotazníku je provedena analýza a návrh aplikace.

        Následuje přehled použitých technologií s krátkým popisem. V další části práce autor popisuje uživatelskou specifikaci aplikace pro dosažení přehledu o funkčnosti a ovladatelnosti systému z internetového prohlížeče.

        V programátoské dokumentaci jsou rozebrány implementační detaily, které stojí za shlédnutí. Dokumentace ve zdrojovém kódu poskytuje dostatek informací pro pochopení a případné rozšíření.

        V závěru práce autor shrnuje dosažené výsledky této práce a přikládá návrhy pro rozšíření, ke kterým je aplikace uzpůsobena.















%%%%%%%%%%%%%%%%%%%%%%%% RESERSE %%%%%%%%%%%%%%%%%%%%%%%%%%%%%%
\chapter{Rešerše stávajících řešení}
\section{Závěrečné práce}

V době průzkumu existujících řešení a zahájení tvorby této práce existovaly v podstatě 3 práce na příbuzné téma. Jedna(TODO najít a citovat název práce) z těchto prací se velmi blížila systému, který si tato práce dává za cíl implementovat. Jako nedostatek se autorovi jevil fakt, že tento zmíněný systém netěží z potřeb veřejnosti a specializuje se pouze na požadavky jednoho klubu.

- TODO tady muzu jeste konkretne rozebrat jednotlivy reseni do detailu (dle meho zbytecne, ale nastrel)

- TODO Jedno bylo v Jave se Stripes, vypadalo celkem pouzitelne, ale vedlo statistiky zapasu, a hracu. Coz je mi prece k nicemu, kdyz o tohle se stara liga, ve ktere soutezim. 
Za timto ucelem by se mohlo navrhnout rozsireni pro tahani dat ze serveru jednotlivych organizaci. O to jsem se uz jednou snazil konkretne u Ceske florbalove unie a narazil jsem na to, ze nebyli ochotni naprogramovat api, ktery by se dalo vyuzivat (stalo by je to penize navic). To by se dalo zminit, ale nemam to vube prozkoumane, takze by mi nekdo mohl vytknout, ze ta a ta org. to api poskytuje.

- TODO Dalsi reseni je postavene na google api. Jde o systemek, ktery umoznuje vytvaret si is v cloudu. Vyhoda toho je, ze ty kluby, ktery maji v tom cloudu IS jsou provazane nejakou funkcionalitou (zpravy, hledani kontaktu, zapasu atp). To je taky neco jineho, nez o co se snazim ja. 

- TODO Treti reseni bylo celkem fajn, uz nevim v cem, ale bylo to zamerene primo na miru jednoho klubu kdesi od Zlina, takze taky neco jineho, nez se snazim ja.

\section{Open source řešení}

Další cestou k alespoň částečné správě vnitřních politik sportovních klubů je volba open source řešení v podobě již předpřipravených redakčích systémů. Příkladem mohou být známé systémy Joomla \footnote{http://www.joomla.org/}, WordPress\footnote{http://www.cwordpress.cz/}. Tyto systémy jsou však vytvořeny s důrazem na jednoduchou aplikaci, takže jejich funkcionalita se ne vždy hodí pro informační systémy tohoto rozsahu a typu. Z tohoto důvodu by se výsledná realizace v těchto systémech stala přinejmenším težkopádnou.

\section{Produkty komerční sféry}
Dle \cite{analyza-sis}, objednávka informačního systému na míru se pro amatérské kluby s nízkým rozpočtem může stát jen velmi těžce zvladatelnou zátěží.

Služby této povahy jsou na našem trhu samozřejmě dostupné. Aby výsledný systém pokryl požadavky, musel by však klub být schopný postrádat nemalý finanční obnos v řádech několika desítek tisíc korun.

\section{Důvody vzniku nového projektu}
Dle \cite{analyza-sis} a vlastního průzkumu koncových amatérských sportovních klubů, existuje malé množství používaných informačních systémů. Dále, dle předchozí sekce a \cite{analyza-sis} je zřejmé, že ideálním řešením je vývoj informačního systému svépomocí v rámci jednotlivých sportovních klubů.

Úsilí vynaložené pro vytvoření systému na míru každému respondentovi z průzkumu, by bylo obrovské. Z tohoto důvodu je na místě vytvořit alespoň nějaký modulárně založený základ informačního systému s tím, že každý uživatel (klub) si může svůj potřebný modul doprogramovat dle svých představ a zveřejněním vyvinutého rozšíření tak stávající systém zdokonalit.










%%%%%%%%%%%%%%%%%%%%%%%%% ANALYZA %%%%%%%%%%%%%%%%%%%%%%%%%%%%%
\chapter{Strukturovaná analýza}

        \section{Tvorba systému požadavků}

                \subsection{Potřeby sportovních klubů}
                Je zřejmé, že individuální potřeby jednotlivých sportovních klubů se budou různit nejen dle prostředí ve kterém fungují, ale i dle povahy a potřeb aktivit, které provozují. Dá se však předpokládat, že řízení a administrativa většiny klubů stojí na velmi příbuzných procesech. Tudíž je možné tyto procesy unifikovat a implementovat v informačním systému, který by výkon těchto, zejména administrativních, činností zpřehlednil, sjednotil a zabezpečil. 

                Úkolem takovéhoto systému však není pokrýt zájmy všech uživatelů, nýbrž vytvořit jakýsi základ, který bude možné dále rozvíjet.

                Svůj předpoklad autor ověřil předložením dotazníků přibližně čtrnácti stům zástupců nezávisle vybraných sportovních organizací v rámci České republiky. Osloveny byly organizace provozující rozdílné aktivity, v naprosté většině však šlo o kolektivní sporty. 

                Předkládaný dotazník (viz příloha TODO LINK?) se skládal ze tří částí:

                \begin{itemize}
                    \item \textbf{Informace o klubu} \newline
                    První část sloužila ke zmapovaní úrovně a velikosti sportovní organizace spolu se způsobilostí respondenta zastupující daný klub.
                    \item \textbf{Informace o aktuálním stavu} \newline
                    Tato část sloužila k odhalení aktuálního stavu informační způsobilosti a potřeb sportovní organizace.
                    \item \textbf{Co by měl informační systém pro sportovní kluby nabízet} \newline
                    V závěrečné části dotazníku autor předložil koncept základního rozvržení potenciálního informačního systému na jednotlivé moduly a tázal se na jejich využitelnost v daném sportovním klubu.
                \end{itemize}

                \subsection{Vyhodnocení dotazníku}
                Výše zmíněný dotazník byl vytvořen za pomoci služby Google forumláře \footnote{http://www.google.com/forms/about/}. Jeho distribuce byla provedena prostřednictvím e-mailové komunikace s příslibem anonymity dotazovaných. 

                Z celkového počtu oslovených svou odpovědí přispělo celkem 221. Relevantnost jednotlivých odpovědí dotazovaných lze ověřit, dle délky působení ve vedoucí pozici svého klubu, v seznamu odpovědí, který je uveden v příloze.\\ TODO LINK

                Kontakty na tyto osoby byly obdrženy na dotaz od kontaktních osob nadřazených organizací, popřípadě z adresářů zveřejněných na jejich webových prezentacích.

                Při průzkumu byl kladen důraz na nehomogenitu sportů, které kluby provozují, proto byly oslovovány kluby z následujícíh sportovních oblastí (s příslušnými nadřazenými organizacemi):

                \begin{itemize}
                \item \textbf{Florbal} (Česká florbalová unie)
                \item \textbf{Volejbal} (Český volejbalový svaz)
                \item \textbf{Basketbal} (Česká basketbalová federace)
                \item \textbf{Házená} (Český svaz házené)
                \item \textbf{Lední hokej} (Český svaz ledního hokeje)
                \item \textbf{Fotbal} (Okresní fotbalová soutěž Šumperk)
                \item Tělovýchovné jednoty (Česká obec sokolská) \footnote{Byly zahrnuty pro svou sportovní univerzalitu.}
                \end{itemize}

                \paragraph*{První a druhá část dotazníku}

                \paragraph*{}

                \noindent
                \textbf{Jakým sportem se váš klub zabývá?}\\
                Z pohledu množství odpovědí na dotazník, z daleka ne však z počtu obeslaných zástupců, se projevila nejaktivněji komunita zabývajícím se florbalem se 71\%.\\

                \noindent
                \textbf{Kolik osob čítá Vaše aktivní členská základna?}\\
                Velikost členské základny sportovních klubů podílejících se na tomto průzkumu byla ze 63\% menší než 100 členů, z 23\% větší jak 100, ale menší než 200 a celých 14\% dotazovaných disponuje členskou základnou o více jak 200 členech. Z tohoto lze usoudit, že obdržené výsledky byly opravdu poskytnuty cílovou skupinou uživatelů navrhovaného informačního systému.\\

                \noindent
                \textbf{Kolik máte soutěžících kategorií}\\
                Při dotazu na počet kategorií, do kterých se členská základna dělí, byla nejčastější odpověď od 1 do 5ti s 31 - 7\%. \\

                \noindent
                \textbf{Cítíte potřebu prezentace Vaší organizace na internetu?}\\
                Další pokládanou otázkou byl dotaz na potřebu prezentace činnosti sportovních klubů na internetu. Odpovědí bylo ze 79\% ano. \\

                \noindent
                \textbf{Má Váš klub vlastního správce webového obsahu?}\\
                Na tuto otázku navazoval dotaz, zda kluby mají člověka, který by se o tuto prezentaci staral a potenciálně tak zastoupil funkcni administrátora systému. Z 66\% byla odpověď kladná.\\

                \noindent
                \textbf{Považujete za výhodné sjednotit komunikaci a organizaci dění v klubu do jednoho celku v
                podobě informačního systému?}\\
                Poslední významnou otázkou z první a druhé sekce byl dotaz na samotnout potřebu vlastního informačního systému. Celých 85\% respondentů tuto potřebu má.\\


                \paragraph*{Třetí část dotazníku}

                \paragraph*{}

                \noindent
                \textbf{Uvítali byste modul evidence členské základny?}\\
                Tato funkcionalita se dá u informačního systému pracujícím s lidskými zdroji považovat za samozřejmou, byla však předmětem dotazu z důvodu pokrýtí zájmu veřejnosti. Výsledný graf procentuálního zastoupení odpovědí lze vidět na obrázku \ref{anal-zakladna}.\\
                \begin{figure}
                \centering
                \mbox{\includegraphics[page=1, scale=0.8]{assets/clenove.pdf} }
                \caption{Graf zájmu o evidenci členské základny \label{anal-zakladna}}
                \end{figure}

                \paragraph*{}
                \noindent
                \textbf{Ocenili byste integraci motivačních prostředků (odměny/pokuty)?}\\
                Při řízení chodu nízkonákladového klubu pracujícího s více než stovkou lidí je třeba zapojit do činnosti i vlastní členy. Z tohoto důvodu se může hodit přítomnost vhodných motivačních prostředků. O výsledcích vypovídá graf \ref{anal-motivace}.\\
                \begin{figure}
                \centering
                \mbox{\includegraphics[page=1, scale=0.8]{assets/motivace.pdf}}
                \caption{Graf zájmu o motivační prostředky \label{anal-motivace}}
                \end{figure}

                \paragraph*{}
                \noindent
                \textbf{Uvítali byste evidenci plateb?}\\
                Další funkcionalitou, která by pomohla usnadnit administrativní činnost vedení klubu by mohla bezesporu být evidence plateb. O názoru veřejnosti na její vhodnost se čtenář může přesvědčit v grafu \ref{anal-platby}.\\
                \begin{figure}
                \centering
                \mbox{\includegraphics[page=1, scale=0.8]{assets/platby.pdf}}
                \caption{Graf zájmu o evidenci plateb \label{anal-platby}}
                \end{figure}

                \paragraph*{}
                \noindent
                \textbf{Uvítali byste správu aktualit?}\\
                Každému sportovnímu klubu, který má zájem o prezentaci své činnosti, by se jistě mohla hodit správa aktualit. Do jaké míry, se opět můžeme přesvědčit na grafu \ref{anal-clanky}.\\
                \begin{figure}
                \centering
                \mbox{\includegraphics[page=1, scale=0.8]{assets/clanky.pdf}}
                \caption{Graf zájmu o tvorbu aktualit \label{anal-clanky}}
                \end{figure}

                \paragraph*{}
                \noindent
                \textbf{Uvítali byste kalendář akcí/událostí?}\\
                Přehledná evidence událostí by mohla být nezbytnou součástí systému řídího chod klubu s téměř každodenním provozem. O názoru respondentů vypovídají data na grafu \ref{anal-akce}.\\
                \begin{figure}
                \centering
                \mbox{\includegraphics[page=1, scale=0.8]{assets/udalosti.pdf}}
                \caption{Graf zájmu o evidenci událostí \label{anal-akce}}
                \end{figure}

                \paragraph*{}
                \noindent
                \textbf{Uvítali byste nástěnky kategorií?}\\
                Další druh notifikace nejen událostí, ale především interních sdělení, mohou reprezentovat tzv. nástěnky. Procentuální zastoupení názoru koncových uživatelů zobrazuje graf \ref{anal-nastenky}.\\
                \begin{figure}
                \centering
                \mbox{\includegraphics[page=1, scale=0.8]{assets/nastenky.pdf}}
                \caption{Graf zájmu o modul nástěnek \label{anal-nastenky}}
                \end{figure}

                \paragraph*{}
                \noindent
                \textbf{Uvítali byste možnost zasílání privátních zpráv?}\\
                Pro zapozdření věci týkající se klubové činnosti z hlediska komunikace by mohla přijít vhod funkcionalita zasílání privátních zpráv mezi jednotlivými uživateli. Analýzu výsledků pro tento dotaz lze shlédnout na grafu \ref{anal-zpravy}.\\
                \begin{figure}
                \centering
                \mbox{\includegraphics[page=1, scale=0.8]{assets/zpravy.pdf}}
                \caption{Graf zájmu o možnost zasílání zpráv \label{anal-zpravy}}
                \end{figure}

                \paragraph*{}
                \noindent
                \textbf{Uvítali byste komunikační fórum?}\\
                Další funkcionalitou z hlediska komunikace mohou být komunikační fóra. Graf \ref{anal-fora} čtenáři může posloužit pro ověření jejich žádanosti.\\
                \begin{figure}
                \centering
                \mbox{\includegraphics[page=1, scale=0.8]{assets/fora.pdf}}
                \caption{Graf zájmu o existenci fór \label{anal-fora}}
                \end{figure}

                \paragraph*{}
                \noindent
                Graficky znázorněnou sumarizaci všech analyticky přínosných odpovědí na otázky předkládaného dotazníku si lze prohlédnout v příloze práce.

        \section{Analýza a návrh}

        Z vyhodnocení dotazníku je zřejmé, že navrhované moduly reflektují potřeby a požadavky dotazovaných respondentů na analyzovaný informační systém. 

        V následující kapitole autor již dle svého uvážení navrhl konkrétní funkcionalitu a její modularitu.Podrobnější funkční principy a jejich dostupnost uživatelským rolím jsou vizualizovány prostřednictvím nástrojů systémové analýzy.

        Z hlediska přístupnosti funkcionality jednotlivých modulů uživatelům je aplikace rozdělena do 4 následujících sekcí.

        \begin{itemize} \label{def-sekce-aplikace}
            \item \textbf{Administrační sekce} - Poskytuje operace pro správu všech definovaných entit a vztahů mezi nimi.
            \item \textbf{Klubová sekce} - Představuje prezentační čast systému pro zdroje vztažené k celému klubu, či jeho dílčím skupinám/kategoriím.
            \item \textbf{Uživatelská sekce} - Slouží pro prezentaci obsahu spojeného čistě s uživatelem aktuální session.
            \item \textbf{Veřejná sekce} - Jedná se o veřejnou část aplikace, dostupnou nepřihlášeným uživatelům systému.
        \end{itemize}

        Díky požadovanému principu modularity je každému modulu umožněno podílet se na obsahu příslušné sekce zcela nezávisle na ostatních částech aplikace. V následujícím textu je funkcionalita každého modulu rozvržena mezi jednotlivé sekce systému, zmíněné v předchozím výčtu.

        %%%% TODO diagramy udělat na přidání příhlášky atd (projít všechny listenery)

            \subsection{Systémový modul}
            Ačkoliv je navrhován modulárně nezávislý systém, musí existovat jakési jádro, ke kterému se jednotlivé moduly budou připojovat, tudíž na něm záviset. K tomuto účelu slouží modul \textit{systémový}. Ten definuje jakousi společnou platformu udávající vlastnosti a procesy všem modulům společné.\\

            Nezbytnou funkcionalitou modulu je:
            \begin{itemize}
                \item Realizace logiky pro inicializaci navigace napříč celou aplikací
                \item Konfigurace jednotlivých komponent, modulů a knihoven třetích stran
                \item Realizace šablon uživatelského rozhraní
                \item Prezentace veřejné části aplikace
                \item Definice společného chování systému (notifikace, komentáře, služby)
            \end{itemize}

            \paragraph*{Administrační sekce}

            Tato část systémového modulu poskytuje CRUD\footnote{Zkratka z anglických názvů operací reate, read, delete, update} operace nad entitami, které tento modul definuje: 
            \begin{itemize}
                \item \textbf{Typ sportu} - Reflektuje typ sportu z reálného světa.

                \item \textbf{Skupina} - Představuje skupinu členů klubu zabývajícím se daným typem sportem. Uživatelé systému jsou do těchto skupin sdružováni.
            
                \item \textbf{Statická stránka} - Představuje jeden ze základních stavebních kamenů systému. Slouží k prezentaci statického obsahu. Je spjata s určitou skupinou.
            \end{itemize}
            
            \paragraph*{Veřejná sekce} 

            Definuje místo pro zobrazení dat nepřihlášenému uživateli. Jejíž obsah je dán z části staticky v rámci základního designu aplikace a z obsahu, který budou jednotlivé moduly dotvářet dle vlastních potřeb. Jsou zde vykresleny například navigační komponenty, aktuality, kontaktní informace, informace o partnerech, informace o kategoriích a jejich sestavách a webové profily uživatelů.

            \paragraph*{Notifikace}

            Systém musí být schopen upozornit uživatele na výskyt následujících událostí:
            \begin{itemize}
                \item Přihlášení do sezóny
                \item Obdržení zprávy
                \item Obdržení platby
                \item Vytvoření účtu
                \item Aktivaci účtu
                \item Deaktivaci účtu
                \item Změně hesla
            \end{itemize}

            \subsection{Bezpečnostní modul}
            Dalším stavebním prvkem základní funkcionality aplikace je modul \textit{Bezpečnostní}. Jeho úkolem je zajišťovat autentizaci a autorizaci uživatelů systému a spravovat vztahy dvojic uživatel,role ke skupinám.

            \paragraph*{Administrační sekce}

            Z hlediska administračního je nezbytná správa autorizačních entit a jejich vzájemných vztahů.
            \begin{itemize}
                \item \textbf{Role} - Představuje zástupnou entitu uživatele pro účely autorizace.
                \item \textbf{Pravidlo} - Nese informaci o pravidle řízení přístupu uživatel k akcím systému.
                \item \textbf{Pozice} - Udává příslušnost uživatele v určité roli v dané skupině.
            \end{itemize}

            \paragraph*{Veřejná sekce}

            Do této sekce \textit{Bezpečnostní} modul přispívá výpisem záznamů z výše zmíněných systémových pozic.

            \subsection{Modul správy uživatelů}
            Správa členské základny, tedy evidence uživatelů neodmyslitelně patří mezi základní funkce informačního systému. 

            \paragraph*{Administrační sekce}

            V této sekci jsou přístupny akce potřebné pro pokrytí kompletní správy členů a entit s nimi spjatých:

            \begin{itemize}
                \item \textbf{Uživatel} - Entita obsahující potřebná data o uživateli (členu klubu). 
                \item \textbf{Kontakt} - Entita nesoucí kontaktní informace na uživatele a případně jeho zákonného zástupce.
                \item \textbf{Adresa} - Podřízená entita \textit{Kontakt}u, nesoucí data o adrese bydliště člena.
                \item \textbf{Profil} - Entita sloužící pro prezentaci \uv{komunitních} informací o uživateli ve veřejné sekci. Její obsah si spravuje každý uživatel sám.
            \end{itemize}

            Z důvodu možnosti volitelné úpravy profilových dat uživatelem, je vhodné umožnit administračním rolím kontrolu tohoto obsahu.

            Další žádoucí funkcionalitou je možnost deaktivace uživatelského účtu, což zabrání jeho přihlášení do systému.

            \paragraph*{Uživatelská sekce}

            V této sekci jsou přístupny akce pro úpravu osobních dat výše uvedených entit a změnu přístupového hesla.

            \paragraph*{Veřejná sekce}

            Veřejnosti přístupno je pouze to, co si uživatel přeje, tudíž data z \textit{Profil}u včetně profilové fotky. Pokud má uživatel zájem zůstat v anonymitě, zobrazuje se jen jeho \uv{přezdívka} a příslušnost v sestavě dané kategorie.

            \subsection{Modul správy sezón}
            Soutěžní období převážné většinu sportů je vymezeno takzvanými sezónami. Správa těchto entit v navrhovaném systému je úkolem tohoto modulu. Jedná se o další z několika základních fukčních bloků.

            \paragraph*{Administrační část}

            V této části jsou obsaženy nástroje pro manipulaci se sezónami a entitami s nimi spojenými:
            \begin{itemize}
                \item \textbf{Sezóna} - Označení časového období (intervalu) v kalendářním roce.
                \item \textbf{Sezónní povinnost} - Nese informaci o výši členských závazků pro určitou skupinu v dané sezóně. Například výše členského příspěvku.
                \item \textbf{Přihláška} - Představuje příslušnost člena klubu k určité skupině pro danou sezónu.
            \end{itemize}

            Vytvořením přihlášky člena do sezóny dojde k plnění závazků uložených v entitě \textit{Sezónní povinnost}. Konkrétně v případě členského příspěvku je modulem plateb vygenerována nová platba a v bezpečnostním modulu dojde k přiřazení člena do pozice hráče v dané skupině.

            \subsection{Modul aktualit}
            Tento modul slouží pro prezentaci dění klubu na veřejnosti.

            \paragraph*{Administrační sekce}

            Sekce ke správě a kontrole fáze publikace jednotlivých novinek realizovaných entitou \textbf{Článek}.

            \paragraph*{Veřejná sekce}

            O prezentaci článků spravovaných tímto modulem se stará modul systémový, který za tímto účelem definuje prostor v rámci domovské stránky systému.

            Další možností, jak přistoupit k obsahu článků je kanál Rss\footnote{Z anglického Rich Site Summary} kanál, který je také v kompetenci modulu aktualit.

            \subsection{Komunikační modul}
            Z komunitního hlediska je vhodná existence komunikačních fór a možnosti zasílání zpráv. Správa těchto zdrojů je umístěna zde.

            \paragraph*{Administrační sekce}

            V rámci této sekce je umožněno manipulovat se stavy komunikačních entit:
            \begin{itemize}
                \item \textbf{Fórum} – Entita reprezentující komunikační fórum.
                \item \textbf{Vlákno fóra} – Entita představující jednotlivá témata v rámci fóra, která jsou dále komentována systémovými komentáři.
                \item \textbf{Zpráva} – Osobní zpráva zasílaná v rámci systému.
            \end{itemize}

            \paragraph*{Klubová sekce}

            V této sekci je uživatelům přístupný přehled fór filtrovaný dle příslušnosti ve skupinách.
            Tvorba témat, neboli \textit{Vláken} je zde umožněna rolím jiným než administračním.

            \paragraph*{Uživatelská sekce}

            Zde jsou zobrazeny osobní zprávy rozdělené na \textit{Přijaté}, \textit{Odeslané}       a \textit{Smazané}. Uživatel zde také může vytvářet zprávy nové. Ty je pak možné odeslat buď jednotlivě nebo hromadně.



            \subsection{Modul událostí}
            Slouží k editaci a přehledné notifikaci o konání klubových událostí.

            \paragraph*{Administrační sekce}

            Úkolem této sekce je tradiční administrace entit v tomto modulu definovaných.           Jedná se o:
            \begin{itemize}
                \item \textbf{Událost} – Představuje samotnou událost konanou v nějakém časovém intervalu. Obsahuje také limit pro úpravy účastí.
                \item \textbf{Účast na události} – Reprezentuje relaci mezi \textit{Uživatelem} a \textit{Událostí}. Účast může být buďto potvrzená nebo zamítnutá. V obou případech je součástí nepovinný komentář.
            \end{itemize}

            \paragraph*{Klubová sekce}

            Zde je dostupný přehled všech klubových událostí a to ve dvou typech zobrazení, kalendář/agenda\footnote{Chronologický seznam}. Stejně jako v případě přehledu \textit{Článků}, či \textit{Fór}, i zde je možná filtrace dle skupin.

            Dále je zde uživatelům umožněna změna své účasti na dané události.

            \paragraph*{Uživatelská sekce}

            V rámci této sekce je zobrazen seznam událostí, u kterých uživatel vyjádřil svoji (ne-)účast.

                \subsection{Motivační modul}
            Motivační modul zavádí do aplikace dvě další entity:
            \begin{itemize}
                \item \textbf{Motivační povinnost} – Definuje předpis výše požadovaných kreditů pro \textit{Skupinu} a danou \textit{Sezónu}.
                \item \textbf{Motivační záznam} – Představuje jednotlivý záznam s výší motivačních prostředků. Dané prostředky mohou být po uživateli požadovány (kredity, pokuty), či přidělovány (odměny) v závislosti na typu tohoto záznamu.
            \end{itemize}

            \paragraph*{Administrační sekce}

           Slouží pro snadnou správu entit motivačního modulu a jejich vztahů k uživatelům systému.

           \paragraph*{Uživatelská sekce}

           Umožňuje zobrazení všech motivačních záznamů přiřazených k aktuálnímu uživateli.
            
            \subsection{Modul partnerů}
            Tento modul poskytuje podporu online marketingu v podobě evidence partnerů a jejich automatické prezentace ve veřejné části systému. Pracuje s entitou \textbf{Partner} – která nese informaci o URL\footnote{Uniform Resource Locator} webu partnera, obrázkovému logu a názvu. Současně poskytuje prostor pro vedení poznámek a možnost deaktivace.

            \paragraph*{Administrační sekce}

            Obsahuje základní CRUD operace pro správu uvedené entity.

           \paragraph*{Veřejná sekce}

            Pro tuto sekci zde existuje komponenta zobrazující logo, odkaz a název všech
            aktivních partnerů klubu.


            \subsection{Modul plateb}
            Tento modul slouží k zadávání příkazů k úhradě jednotlivým uživatelům. Zavádí entitu \textbf{Platba}, která obsahuje všechny informace potřebné ke bezpečnému provedení a identifikaci platby.

           \paragraph*{Administrační sekce}

           Zde je dostupná správa všech plateb. Zadanou platbu lze smazat pouze pokud ji už uživatel neoznačil jako odeslanou.
           
           Po vytvoření \textit{Přihlášky} k sezóně je přihlašovanému uživateli připsána platba s výší členského příspěvku. Tuto funkcionalitu lze v nastavení aplikace zablokovat.

           \paragraph*{Uživatelská sekce}

           Uživateli je zobrazen seznam všech plateb k němu přiřazených. Je zde dostupná funkce označení platby za odeslanou.            

                \subsection{Modul nástěnek}
            Tento modul zanáší do aplikace možnost vyvěšování informačních příspěvků v rámci takzvaných \uv{Nástěnek}. Definuje entitu \textbf{Příspěvek} – Ta reprezentuje příspěvek na nástěnce dané skupiny.

           \paragraph*{Administrační sekce}

            Poskytuje administrační rozhraní pro správu příspěvků nástěnek jednotlivých skupin.

           \paragraph*{Klubová sekce}

           Slouží pro přehlednou prezentaci informačních příspěvků.
           Každá nástěnka je abstraktní entitou obsahující jednotlivé \textit{Příspěvky}. Ty se dále dělí dle stavu do sekcí \uv{Sdělení a varování}, \uv{Aktivní příspěvky} a \uv{Historie nástěnky}. Dále je zde dostupná filtrace nástěnek dle požadované skupiny.


            \paragraph*{Vstupy systému}

            Veškerá data jsou do systému zadávána ručně prostřednictvím webového rozhraní ve webovém prohlížeči nebo automatizovaně v případě dat z konfigurace.

            \paragraph*{Výstupy systému}

            Hlavní datový výstup je realizován opět zobrazením ve webovém prohlížeči. Dalšími výstupními médii jsou notifikační emaily, logovací zprávy a exportované \verb|*.csv| soubory. 

            \paragraph*{Datová úložiště}

            Naprostá většina zpracovávaných dat bude uložena v rámci databázového systému na serveru.
            Ostatní, například obrazová, data budou uložena v adresáři \verb|assets/|.


            \paragraph*{Studie případu užití}


            \begin{figure}
                \centering
                \mbox{\includegraphics[page=1, scale=0.6]{assets/use_case.pdf}}
                \caption{Diagram případu užití \label{use-case}}
            \end{figure}

            - vytvorit use case modely tech entit, kde hraji roli listenery\\
                -- registrace uzivatele\\
                -- prihlaseni\\
                -- prihlaseni na udalost\\
                -- udeleni kreditu\\
                -- kontrola weboveho profilu\\
                -- zmena hesla\\
            -- navrh datoveho modelu class diagram\\
            -- activity model\\

        \section{Návrh řešení implementace}
        V této sekci je podrobněji popsán soubor softwarových komponent, které systém potřebuje ke svému běhu. Dále je zde představeno několik hlavních nástrojů použitých během vývoje aplikace.
        -- takze se tu bude mluvit o tom jak se to naprogramuje, vlastne popisu zpusoby, ktere jsem nakodil\\

                \subsection{Zhodnocení možností použití existujících přístupů}
                - Takže musím zhodnotit pro a proti stávajících nejvíce používaných přístupů a jazyků pro tvorby informačních systému.

                Nejvice se pouzivaji asi Java C\# PHP RUBY PERL
                - Porovnám jejich pro a proti. Jaký přístup se k čemu hodí.
                - Například to, že se musíme seznámit a zaměřit na velikost a rozsah nasazení cílených systémů.
                - Zvážit k čemu se jaký přístup hodí, musíme brát v potaz i velikost tvůrčího týmu. Na prostředí, ve kterém vyvíjený systém bude nasazen a provozován. Zmínit i vliv na rychlost vývoje projektů v závislosti na použitém přístupu. Zvážit bezpečnost vývoje v závislosti na přístupu. Zaměřit se na vstupní bariéry a učící křivky přístupů, jazyků a frameworků na nich postavených. Zmínit výhody php, naťuknout jeho vývoj a postupné obohacování a nasazování standardizovaných technik ze silnějších jazyků. 

%% http://cs.wikipedia.org/wiki/Informa%C4%8Dn%C3%AD_syst%C3%A9m#Aktu.C3.A1ln.C3.AD_trendy
%%http://cs.wikipedia.org/wiki/Java_EE

%%https://dspace.k.utb.cz/bitstream/handle/10563/5897/huspenina_2008_dp.pdf?sequence=1

%% http://is.muni.cz/th/172898/fi_b/
\paragraph*{Proc se ty borce java tak libi}

.
Jako vhodný programovací jazyk pro implementaci systému se mi jeví
Java a to z nˇekolika následujících d ˚uvod ˚u. Java je objektovˇe orientovaný
programovací jazyk nezávislý na platformˇe. Je tedy možné provozovat
systém prakticky v jakémkoliv operaˇcním systému. Dálší pˇredností Javy
je její kvalitní a rozsáhlá dokumentace. Souˇcástí Java 5 EE je i mimo jiné
technologie JavaServer Faces, která umož ˇnuje rychlý vývoj profesionálních
webových aplikací.
Jelikož informaˇcní systém obsahuje d ˚uvˇerná data o klientech a jiná d ˚uležitá
data, jeho bezpeˇcnost by mˇela být na prvním místˇe. Systém by mˇel
uživatele pˇristupujícího k systému autentizovat (bezpeˇcnˇe zjisit identitu
pˇrihlašovaného) a autorizovat (udˇelit urˇcitá pˇrístupová práva). Pˇri pou-
žití programovacího jazyku Java lze využít plnˇe integrované služby Java
Authentication and Authorization Service.
Systém by mˇel být kompatibilní s nˇekterými aplikacemi bˇežnˇe použí-
vanými v advokátních kanceláˇrích jako jsou napˇríklad systémy pro práci
s právními informacemi (ASPI, Codexis) a SymbioASPI (viz 3.2.2).


       \paragraph*{Proč je PHP tak oblíbené?}
-PHP je relativně jednoduché na pochopení\\
-PHP má syntaxi velmi podobnou jazyku C a je tedy většině vývojářů dost blízký\\
-PHP podporuje širokou řadu souvisejících technologií, formátů a standardů\\
-je to otevřený projekt s rozsáhlou podporou komunity\\
-dají se najít kvanta již hotového kódu k okamžitému použití nebo funkční PHP aplikace. Podstatná část z hotového kódu je šířena pod nějakou svobodnou licencí a dá se použít ve vlastních projektech\\
-PHP si dobře rozumí s webovým serverem Apache (aby ne, vždyť je to sesterský projekt spravovaný Apache software foundation)\\
-PHP snadno komunikuje s databázemi, jako je MySQL, PostgreSQL a řada dalších\\
-PHP je multiplatformní a lze jej provozovat s většinou webových serverů a na většině dnes existujících operačních systémů\\
-PHP podporuje mnoho existujících poskytovatelů webhostingových služeb\\
       \paragraph{PHP nevýhody?}
       -PHP je interpretovaný, ne kompilovaný jazyk\\
-kdokoli má přímý přístup k serveru, může nahlédnout do vašich PHP skriptů\\
-Podpora objektového programování není v PHP na moc dobré úrovni. V PHP 5 by se to ale mělo zlepšit.\\
-protože je PHP aktivně vyvíjen, v budoucích verzích jazyka se mohou některé funkce změnit nebo se mohou chovat jinak než dosud. \\

                Vystup toho by mel byt, ze jsem se rozhodl pro PHP.

                \subsection{Apache 2}
                Co je to?
                Nastavení?

                \subsection{Databázový server MySQL/Postgres}
                Co je to?
                Nastavení?

                \subsection{Databázová vrstva}
                Zde uvést notORM a ORM veci a proc jsem si zvolil ORM
                Prehled dostupnych veci zase s vystupem proc ORM.

                \subsection{Nette framework}
                Co je to?
                Nastavení? - potřebné moduly?
                -- sem uvest popis veci, ktere se vyuzivaji v nette, ze jsou tam dodany rozsirenima je vec az dalsich pouzitych veci
                -- proste popis vyuzitych principu v ramci frameworku
                -- nette compiler extensions\\
                -- udalosti\\


                Tahle část se podobá sekci použitých technologií v implementační části dokumentu.


























%%%%%%%%%%%%%%%%%%%%%%%%%%%%% WEBDESIGN %%%%%%%%%%%%%%%%%%%%%%%%%%%%
\chapter{Webdesign}

Jelikož předmětem této práce je tvorba webového informačního systému ...
- kratky vyvoj webu na desktopech odkud se weby vzaly
- definovat pojem webdesign, pac se dale zminuje
- dal by se hodilo, co to webova aplikace je
- zminit trendy
- plynule prejit na web x.0

\section{Web 1.0}
Je to počátek webu. Nic moc prostě jen statický stránky.
- pár odstavců k historii na téma web1.0
- BP

\section{Web 2.0}
Web jak ho známe uvést příklady.
- pár odstavců k historii web2.0 
- pretlumocit odstavec z BP

- zminit rozmach mobilnich zarizeni,

\section{Web na mobilních zařízeních}
Co to vůbec je?

- masivní přechod na mobilní zařízení a potřeba aplikací držet krok s rozvojem
- zjistit neco o desihnu na mobilnich zarizenich

\section{Responsivní webdesign}
Jak to souvisí s webem na mobilních zařízeních.
- zde troška teorie o tom, co to vlastně je ten responsivní webdesign


%%% kdyz to cele bude tak na 3 strany tak je to ok, min asi ne



















%%%%%%%%%%%%%%%%%%%%%%%%%%%%%% IMPLEMENTACE %%%%%%%%%%%%%%%%%%%%%%%%%%
\chapter{Implementace}

        \section{Testování}
        - zde napsat odstavec o dostupných technologiích pro testování v php
        - nevim zda ke kazde psat co umi a v cem je dobra, to je asi zbytecne
        - predstavim Nette \textbackslash Tester a popisu pouziti na ukazkach primo z aplikace
        TODO

        \section{Nasazení}
        - Něco o nasazení aplikace
        - co sem kruci???

        \section{Použité technologie a knihovny}

        \subsection{PHP}
        PHP 5.5 
        PHP je velmi rozšířený skriptovací jazyk využívaný pro tvorbu webových aplikací. Vznik se datuje rokem 1994, kdy byl původní systém napsán v jazyce Perl a pozdějí přepsán v jazyce C. Jeho první oficiální verze byla PHP/FI 2.0, v době psaní této práce je aktuální verze 5.6.3. Díky podobné syntaxi s jazykem C, získalo PHP na oblibě u mnoha programátorů a dle \ref{php-lin} se stalo nejčastěji používaným modulem serveru Apache. Jelikož se jedná o skriptovací jazyk, je potřeba, aby na serveru běžel interpretr.

        Minimalni konfigurace PHP pro běh Nette frameworku:
        \begin{itemize}
            \item Povolené funkce \verb|error_reporting()| a \verb|flock()|
            \item Direktiva \verb|Variables_order| musí být povolena
            \item Direktiva \verb|Register_globals| musí být zakázána
            \item PCRE\footnote{Perl COSI Regular Expression} rozšíření musí podporovat UTF-8
            \item \verb|Reflection phpDoc| musí být povoleno z důvodu využívání anotací
            \item Rozšíření \verb|iconv| musí být aktivní
            \item Rozšíření PHP tokenizer musí být aktivní
            \item Multibyte String function overloading ON
            \item Musí být dostupné proměnná \verb|$_SERVER["HTTP_HOST"]| nebo\\ \verb|$_SERVER["SERVER_NAME"]|
            \item Musí být dostupné proměnná \verb|$_SERVER["REQUEST_URI"]| nebo \verb|$_SERVER["ORIG_PATH_INFO"]|
            \item Musí být dostupné proměnná \verb|$_SERVER["SCRIPT_NAME"]| nebo \verb|$_SERVER["DOCUMENT_ROOT"]|
            \item Musí být dostupné proměnná \verb|$_SERVER["REMOTE_ADDR"]| nebo funkce\verb|php_uname("n")|
        \end{itemize}

        \subsection{Apache HTTP server}
        Jedná se o multiplatformní webový server sloužící pro odbavování požadavků protokolu HTTP\footnote{HyperText Transfer Protocol} z/do aplikací v rámci tohoto serveru obsluhovaných. Apache v současnosti podporuje jazyky jako je například Python, PHP, Perl, či Tcl. Mnoho funkcí podporovaných tímto serverem je implementováno v podobě kompilovaných modulů rozšiřujících jádro. 
        Tyto moduly se aktivují přesunutím příslušných \verb|*.conf| a \verb|*.verb| souborů z adresáře \verb|<apache>/mods-available| do adresáře \verb|<apache>/mods-enabled|, kde \verb|<apache>| je instalační adresář webového serveru.

        Příklad modulů, aktivovaných při vývoji této práce jsou:
        \begin{itemize}
        \item \verb|mod_rewrite| -- Poskytuje systém pro přepisování požadovaných URL za běhu aplikace. 
        \item \verb|php5_module| -- Poskytuje podporu pro jazyk PHP.
        \item \verb|mime_module| -- Poskytuje mapování různých částí souborového systému ve stromu dokumentu a pro přesměrování.
        \item \verb|xdebug_module| -- Umožňuje ladění výkonu skriptů aplikace.
        \end{itemize}

        Při vývoji této diplomové práce byl použit HTTP server Apache verze 2.4.10. Dále je vhodné poznamenat, že vývoj této práce byl prováděn na linuxové distribuci Ubuntu 12.04.5 LTS, tudíž uvedené postupy konfigurací se mohou v jiných operačních systémech lišit. 

        \subsection{Nette framework}
        \paragraph*{}
        Nette framework je skriptovací rámec sloužící pro tvorbu webových aplikací a služeb v dynamicky typovaném jazyce PHP 5. Hlavním cílem tohoto frameworku je eliminace co největšího počtu rizik s tvorbou webových aplikací spojených, a poskytnutí prostředí pro vývoj efektivního, avšak čistého kódu. 

        Z hlediska architektury frameworku je nesporným pozitivem jeho promyšlený návrh, ve kterém se částečně odráží již známe konvence například z EJB 2.0. Framework se také pyšní využíváním nových vlastností PHP 5 a na událostech založeném modelováním komponent \cite{nette}. 

        Vnitřní struktura frameworku využívá návrhového vzoru MVP\footnote{Model View Presenter}, který je odvozen od známého vzoru MVC\footnote{Model View Controller}. 

        Důležitou vlastností frameworku je nezávislost jeho vnitřních komponent, jako jsou například \verb|Nette\Forms|, \verb|Nette\Tester| či šablonovací systém \verb|Latte|.

        Z hlediska zmíněné bezpečnosti framework poskytuje ochranu proti CSRF\footnote{Cross-Site Request Forgery}, URL útok, Session hijacking, session stealing, session fixation a případě XSS\footnote{Cross-Site Scripting} přichází s vlastním řešením Context-Aware Escaping.

        \subsection{Doctrine 2}
        \paragraph*{}
        Doctrine 2 je ORM\footnote{Object-Relational Mapping} framework pro jazyk PHP verze 5.3. a vyšší. Za použití objektového přístupu dosahuje vysoké míry abstrakce nad použitou relační databází. Čímž uživateli umožňuje přistupovat k datům jako ke konzistentním objektům. V porovnání s běžně používanými \verb|HashMap|ami, jako tomu je například u databázové vrstvy Dibi\cite{dibi}, se jedná o užitečný krok vpřed.

        Doctrine 2 framework je do velké míry inspirován Hibernate frameworkem, určeným pro jazyk Java. Jak z hlediska definice entit, tak z hlediska zajištění funkcionality.

        Podobně jako v Hibernate lze entity definovat pomocí komentářových anotací, či v případě potřeby XML nebo YAML zápisem.

        Doctrine 2 vychází z návrhového vzoru DataMapper oproti jiným ORM nástrojům, které stací na návrhovém vzoru Active Record. Dle vzoru DataMapper entita neobsahuje persistentní operace. O tuto funkcionalitu se stará externí objekt EntityManager, který pracuje jako fasáda mezi zbyvající logikou aplikace.

        Z povahy ORM nástroje je Doctrine 2 nezávislá na použité databázi. V součastnosti jsou podporovány databázové systémy MySQL, Oracle, PostgreSQL a SQLite.

        Jednou z klíčových vlastností Doctrine 2, mimo mapování objektů, je existence vlastního dotazovacího jazyka DQL\footnote{Doctrine Query Language}. Ten je velmi podobný jazyku HQL\footnote{Hibernate Query Language} z již zmíněného Hibernate frameworku. Tvorba dotazů pomocí DQL nám umožnuje dosáhnout úplné nezávislosti na používaném databázovém systému.

        V neposlední řadě je potřeba zmínit, že u databází, které podporují transakce je Doctrine využívá zcela automaticky. Ne příliš známou výhodou je vnitřní podpora stromových struktur.
        \paragraph*{}
        Celý framework se skládá ze tří vrstev:
        \begin{itemize}
        \item Common -- Základní třídy a knihovny pro práci s kolekcemi, událostmi, anotacemi, keší, apod. jsou umístěny právě v této vrstvě. Celá \verb|Common| vrstva je definována ve jmenném prostoru \verb|DoctrineCommon|.
        \item DBAL\footnote{DataBase Abstraction Layer}
            -- Tato vrstva abstrahuje aplikaci od konkrétního databázového systému. Je zde zavedeno DQL. Veškerá logika této vrstvy je definována ve jmenném prostoru \verb|DoctrineDBAL|.
        \item ORM -- Zde je uložena veškerá logika zajišťující mapování objektů, persistenci dat a samozřejmě jejich načítání. Jmenný prostor této vrstvy je \verb|DoctrineORM|.
        \end{itemize}

        % TODO k tomu napsat nejaky citace a zavery of Fowlera
        % TODO K cemu se to nehodi
        %https://ondrej.mirtes.cz/doctrine-2-neni-pomala

        \subsection{Kdyby\textbackslash Doctrine}
        \paragraph*{}
        \verb|Kdyby\Doctrine| (dále jen KD) je rozšíření pro ..
        https://github.com/Kdyby/Doctrine

        \subsection{Kdyby\textbackslash Events}
        \paragraph*{}
        \verb|Kdyby\Events| (dále jen KE) je rozříření kompileru, které do Nette frameworku dodává robustní systém řízení událostí. Funkcionalita KE je postavena na systému událostí z projektu Doctrine a jednoduchém systému událostí, na který se napojíme poděděním \verb|Nette\Object|.

        \paragraph*{}
        Systém událostí v Nette frameworku funguje tím způsobem, že každá \verb|public| vlastnost třídy začínající na \uv{on} je událostí. Registrace handleru se pak provede přiřazením callbacku do zmíněné vlastnosti. Vyvoláním události jako callbacku se pak provedou všechny handlery zaregistrované pro danou událost \cite{nette}. Typickým příkladem běžně používaných událostí v Nette frameworku je zpracování formulářů.
        \paragraph*{}
        Třída frameworku by mohla vypadat nějak takto:\\
        \verb|class Form extends Container {|
        \newline\verb|  |
        \newline\verb|  public $onSubmit;|
        \newline\verb|  |
        \newline\verb|  public function __construct(...) {|
        \newline\verb|      $this->onSubmit = [];|
        \newline\verb|  }|
        \newline\verb|}|
        \paragraph*{}
        V kódu aplikace pak:
        \newline\verb|$form = new Form();|
        \newline\verb|$form->onSubmit[] = function ($form) {|
        \newline\verb|      $values = $form->getValues();|
        \newline\verb|      };|
        \paragraph*{}
        V metodě zpracování formuláře vyvoláme událost:\\
        \noindent\verb|$this->onSubmit($this);|
        \paragraph*{}
        Následně jsou všechny metody registrované pro danou událost třídou \verb|Nette\Object| iterovány a je jim předán definovaný parametr.

        Události v systému Doctrine fungují na principu vytvoření listeneru, implementujícího \verb|Kdyby\Events\Subscriber|, zachycované události jsou vraceny vlastní metodou \verb|getSubscribedEvents()|.

        Tento listener musí být zaregistrován v EventManageru, pomocí jehož metody \verb|dispatchEvent()| je příslušná událost vyvolána.

        \paragraph*{}
        \verb|KE| zavádí registraci listenerů jako běžných služeb. Tudíž stačí definice v sekci \verb|services|  konfiguračního souboru. Nezbytností je však uvedení tagu \verb|kdyby.subscriber|, což vede k automatickému zaregistrování listeneru do EventManageru z projektu Doctrine. Při automatické registraci je u všech listenerů kontrolováno, zda rozšiřují \verb|Nette\Object| pro případ, že by obsahovaly některé \uv{Nette události}. Ty jsou v tom případě nahrazeny instancemi třídy \verb|Kdyby\Events\Event|. 
        \paragraph*{}
        \verb|Kdyby\Events\Event| se chovají jako běžné pole, takže Nette \uv{nepozná}, že vyvolává globální událost nad celou aplikací. Ta je propagována do EventManageru, starajícího se o obsluhu všech listenerů.

        \subsection{Kdyby\textbackslash Console}
        \paragraph*{}
        \verb|Symfony\Console| (dále jen SC) je komponenta Symfony frameworku\cite{symfony} s vysoce konfigurovatelnými vlastnostmi, která slouží pro snadnou tvorbu příkazů ovládajících php aplikaci z příkazové řádky. Výstupům příkazů je pak dán přehledný formát.  

        \verb|Kdyby\Console| (dále jen KC) využívá skutečnosti, že komponentu \verb|SC| je možné používat samostatně a integruje ji do \textit{Nette} frameworku v podobě rozšíření kompilátoru\footnote{Nette\textbackslash DI\textbackslash CompilerExtension}.
        \paragraph*{}
        Velmi důležitou vlastností \verb|KC| je snadno použitelné rozhranní pro tvorbu vlastních příkazů.

        Základním krokem je vytvoření vlastní třídy příkazu poděděním třídy Command\footnote{Symfony\textbackslash Component\textbackslash Console\textbackslash Command\textbackslash Command}. Metoda \verb|execute()| pak implementuje vykonávané úkony. Dalším fází registrace nového příkazu je jeho přidání jako služby v příslušné sekci konfiguračního souboru aplikace. 
        \paragraph*{}
        Výše zmíněné \verb|Kdyby\Doctrine| funkcionality tohoto doplňku využívá například při validaci databázového schématu.

        \paragraph*{}
        Jako příklad autor uvádí některé z užitečných příkazů, často používaných při vývoji této diplomové práce:\\
        \verb|php www/index.php \orm:schema:up --force|
        \newline\verb|php www/index.php \orm:schema --validate|

        \subsection{Kdyby\textbackslash Monolog}
        \paragraph*{}
        Monolog je velice robustní knihovna pro logování událostí v PHP verze 5.3 a vyšší, jejíž vývoj vede Jordi Boggiano. Umí posílat logované zprávy do souborů, socketů, databáze a různých webových služeb. Pro každý druh výstupu Monolog definuje vlastní \uv{handler} a \uv{formatter} pro výstupní formát logované zprávy. 

        Monolog disponuje osmi úrovněmi logovaných zpráv, které definuje RFC 5424\footnote{http://tools.ietf.org/html/rfc5424}.

        Rozšíření \verb|Kdyby\Monolog| umožňuje bezproblémovou integraci do frameworku Nette a její standartní instalaci prostřenictvím composeru jako běžné Nette rozšíření.

        V této diplomové práci je použita knihovna verze~1.0.1.

        \subsection{Kdyby\textbackslash Translation}
        \paragraph*{}
        Rozšíření, integrující robustní překladový systém Symfony\textbackslash Translation do Nette frameworku.

        Disponuje třemi možnostmi získání identifikátoru lokalizace.
        \begin{itemize}
        \item Ze Session úložiště
        \item Z locale parametru HTTP požadavku
        \item Z Language hlavičky HTML dokumentu
        \end{itemize}

        Texty s překlady jsou standartně hledány v adresářích \textit{lang/} a \textit{locale/}. Případnou změnu je třeba zadefinovat pod klíčem \verb|dir| v příslušné sekci konfiguračního souboru. Tato direktiva však umožňuje zadefinovat jen statické cesty.

        Pro modulárně orientovaný systém, s předem neznámými adresáři s překlady, nabízí toto rozšíření možnost implementace rozhranní\\ \verb|Kdyby\Translation\DI\ITranslationProvider|, což zajistí dodání potřebných cest v době překladu zdrojových kódů aplikace.

        \verb|Symfony\Translation| umožňuje definovat překladové texty v souborech běžných typů\footnote{php, yml, csv, dat, ini}. \verb|Kdyby\Translation| však přidává navíc možnost definice textů v \uv{Nette vlastním} \textit{neon}\footnote{http://ne-on.org/} formátu.

        Instalace se provádí jako v případě běžného rozšíření v konfiguračním souboru, rozšíření definuje službu \verb|Kdyby\Translation\Translator| a její metoda \verb|translate()| pak přeloží požadovaný klíč dle odpovídajícího \textit{locale} klíče.

        \subsection{Composer}
        \paragraph*{}
        Composer je multiplatformní nástroj pro správu závislostí v PHP projektech. Ve standartním nastavení s balíky a knihovnami pracuje na projektové úrovni, což znamená, že deklarované závislosti instaluje do příslušného projektového adresáře (\textit{vendor/}). 

        Výňatek obsahu deklaračního souboru \textit{composer.json} z tohoto projektu:\\
            \verb|{|
                \newline\verb|"require": {|
                \newline\verb|"php": ">= 5.3.7",|
                \newline\verb|"nette/nette": "2.2.3",|
                \newline\verb|"kdyby/doctrine": "v1.1.0"}|
                \newline\verb|}|.

        K instalaci takto deklarovaných závislostí pak stačí jen nechat provést konzolový příkaz \verb|composer install| z projektového adresáře. Pokud bychom chtěli přidat další závislost, musíme ji vepsat do deklaračního souboru nebo opět z konzole nechat provést: \\ \verb|composer require <název_balíku>:<požadovaná_verze>|.

        \paragraph*{}
        Další významnou dovedností tohoto nástroje je generování tak zvaného \uv{autoloaderu}, tedy třídy která je schopná automaticky načíst všechny třídy, které composer stáhl. K začlenění do projektu stačí pouze přidat řádek \verb|require 'vendor/autoload.php';| do \verb|bootstrap.php| souboru našeho projektu.

        \paragraph*{}
        Composer pro svůj běh vyžaduje PHP verze 5.3.2 a vyšší. V závislosti na typu verzovacího nástroje deklarovaných závislostí je potřeba mít nainstalovaný i tento příslušný verzovací nástroj. V této aplikaci byl použit composer verze \textbf{1.0-dev}. Závislosti pro tento projekt byly vyhledávány na serveru Packagist\footnote{https://packagist.org/}.

        \subsection{Git}
        \paragraph*{}
        Git je velmi rychlý distribuovaný nástroj pro správu verzí souborů. Vznikl roku 2005 a jeho autorem je Linus Torvalds, který git začal vyvíjet původně pro vývoj jádra operačního systému Linux.
        \paragraph*{}
        Jeho odlišností od zaběhlých verzovacích nástrojů jako jsou například Subversion\footnote{https://subversion.apache.org/}, či Mercurial\footnote{http://mercurial.selenic.com/}, je způsob ukládání jednotlivých verzí. Git si pro každou verzi ukládá snímek verzovaného souboru, jehož odkaz vloží do dané verze. V případě, že na verzovaném souboru nebyla provedena změna, uloží Git již existující znímek, čímž se zvýší rychlost a sníží paměťová náročnost. Ke zvýšení rychlosti práce Gitu přispívá fakt, že téměř každá operace je prováděna lokálně. Vliv zpoždění přenosu dat na síti je proto takřka nulový.
        \paragraph*{}
        Jak už bylo zmíněno Git pracuje na lokálním repozitáři, jehož obsah může být vyzrcadlen na vzdálený server. Samotné verze souborů jsou vždy uloženy v nějaké větvi, kde jedna je vždy hlavní. Na lokálním repozitáři je tato větev označena ukazatelem označeným jako \verb|master|, na vzdáleném repozitáři pak \verb|origin master|. Poslední snímek je navíc označen ukazatelem \verb|HEAD|.
        \paragraph*{}
        Jelikož je Git velmi komplexním nástrojem jehož popis vydá za samostatnou knihu\cite{progit}, omezuje se autor pouze na ukázku nezbytného minima pro základní verzování. Je nutné podotknout, že volané příkazy je nutné provádět nad kořenovým adresářem repozitáře, kde byla provedena inicializace.
        \noindent
        \verb|git init|
        \newline\verb|git add config.neon|
        \newline\verb|git commit -m "Iniciální commit"| 
        \newline\verb|git commit -m "Změna konfiguračního souboru"|
        \newline\verb|git push origin master|
        \newline

        Při vývoji této diplomové práce byly zdrojové kódy spravovány nástrojem Git verze~1.7.9.5. a věřejně dostupné skrze webovou službu GitHub\footnote{https://github.com}.

        \subsection{Javascript}
        \paragraph*{}
        Javascript je dynamický, objektově orientovaný skriptovací jazyk. Byl navržen pro zavedení vetší interaktivity webových prezentací. Je interpretován na straně uživatele ve webovém prohlížeči.

        \subsection{AJAX}
        \paragraph*{}
        Ajax (\textit{Asynchronous JavaScript and XML}) je označením pro použití několika technologií s cílem dosažení asynchonního přenosu dat bez nutnosti znovunačtení celého obsahu HTML\footnote{Hypertext Markup Languge} dokumentu. Jedná se zejména o Javascript, DOM\footnote{Document Object Model} a XML\footnote{Extensible Markup Language}. Asynchronní přenos dat na pozadí aplikace je zajištěn prostřednictvím Javascriptového objektu \verb|XMLHttpRequest|.

        \subsection{Bootstrap framework}
        \paragraph*{}
        Bootstrap je open-source framework pro tvorbu responsivního uživatelského rozhraní webových aplikací založený na jQuery knihovně. Bezespornou výhodou je strmá křivka učení zapříčiněná kvalitní dokumentací a nízkou náročností v začátcích. V nabídce tohoto frameworku nejde jen samotné CSS, ale disponuje i širokou škálou JavaScriptových doplňků. 

        Za pomoci \uv{css media dotazů} umožňuje přizpůsobit design aplikace velikosti různých zobrazovacích zařízení, jako jsou tablety, telefony a laptopy.

        Základ Bootstrapu tvoří kompilované a minimalizované soubory (bootstrap.*, bootstrap.min.*), které je možné použít ve kterémkoliv webovém projektu. 

        Bootstrap využívá dva velmi populární preprocessory Less\footnote{http://lesscss.org/} a Sass\footnote{http://sass-lang.com/}. Grafika je komprimována pomocí ImageOptim\footnote{https://imageoptim.com/}.

        \section{Standardy}
        \subsection{HTML5}
        \paragraph*{}
        HyperText Markup Language je jazyk pro tvorbu hypertextových dokumentů. Od svého vzniku (1991) dostál mnoha změn a~rozšíření. Jeho nejdelé používanou verzí je HTML~4.01, jehož nástupcem měl být standard XHTML~\footnote{Extensible HyperText Markup Language}. Pro široký nesouhlas byl však znovu spuštěn vývoj další HTML verze s číslem 5, která je nyní (2014) nejvyšší verzí specifikace HTML. Verze 5 vyřadila zastaralé prvky a~přináší některé nové. Jako jsou například sémantické prvky, syntaktické prvky v~podobě nových tagů, či nativní podporu pro multimédia nebo offline webové aplikace použitím lokálního úložiště~\cite{html5-all}~\cite{html5-new}.

        \subsection{CSS3}
        CSS je zkratkou pro Cascading Style Sheet, značkovací jazyk pro popis vzhledu a formátování HTML dokumentů. Nejčastěji se používá pro X/HTML webové stránky, ale je aplikovatelné také na XML dokumenty včetně čistého XML, SVG, XUL. 

        Prvotní iterace tohoto standardu vyšla v roce 1996 z Cascading HTML Style Sheets (CHSS), následovaná CSS2 (1998) a CSS2.1 (2005). \uv{CSS3} pak značí poslední verzi CSS disponující rozsáhlejšími možnostmi.

        Společným problémem dřívějších verzí byla komplexita specifikace, která se obtížně rozšiřovala.

        CSS3 přichází s novou koncepcí modulární struktury, rozdělení do více souborů (modulů), díky čemuž umožňuje definovat funkcionalitu odděleně a lépe ji tak spravovat. Těchto modulů má CSS3 zatím něco přes 40, patří mězi ně například Selektory, Jmenné prostory, Barevné a media dotazy. O standardizaci a vývoj CSS3 se stará W3C\footnote{WorldWideWeb Consorcium}, což je jedna z mnoha standardizačních organizací, která zajišťuje internet takový, jaký ho známe.

























%%%%%%%%%%%%%%%%%%%% UZIVATEL %%%%%%%%%%%%%%%%%%%%%%%%
\chapter{Uživatelská dokumentace}
V následující kapitole je podrobně rozebrána dokumentace přístupu přes webový prohlížeč. Každý modul popisuje svoji strukturu a možnosti v rámci sekcí, do kterých zasahuje.

    \section{Struktura aplikace}
        V první řadě je důležité si představit základní strukturu rozhraní aplikace (dále jen rozhranní). 
        To je rozděleno na tři základní části:
        \begin{itemize}
            \item \textbf{záhlaví}
            \item \textbf{tělo stránky}
            \item \textbf{zápatí}
        \end{itemize}

        \subsection{Záhlaví}

            Obsahuje logo klubu s odkazem na \textit{Domovskou stránku}, prostor pro rozbalovací navigační nabídky jednotlivých sekcí (dále jen \uv{horní menu}), formulář pro vyhledávání ve článcích a veřejnou nabídku (dále jen \uv{veřejné menu}).

            \paragraph*{Horní menu}

            Data zobrazená na této pozici závisí na stavu autentizace sezení. V případě, že uživatel není přihlášen, je zde vypsán udaj o tomto stavu spolu s tlačítkem \label{login-link}\uv{Přihlásit} vedoucím na stránku s přihlašovacím formulářem. Opačně pak v případě, že přihlášen je, se zde objeví rozbalovací nabídky skrývající navigaci pro každou sekci zvlášť.

            \paragraph*{Veřejné menu}

            Tuto navigaci vidí všichni uživatelé aplikace bez ohledu na to, zda jsou přihlášeni, či nikoliv.
            Zde jsou zobrazovány odkazy do veřejných sekcí jednotlivých modulů. Dále je zde, již nezávisle na modulech, zobrazen výpis všech kategorií s jejich informačními stránkami.

        \subsection{Tělo stránky}
            Nejvíce dynamická rozhranní, kde jsou prezentována veškerá data. Tato část rozhraní se dále dělí na tři podčásti:
            \begin{itemize}
                \item \textbf{Levá navigace}
                \item \textbf{Drobečková navigace}
                \item \textbf{Notifikace}
                \item \textbf{Obsah}
            \end{itemize}

            \paragraph*{Levá navigace}

            Nejedná se jen o jednu navigační nabídku. V této části se mohou objevit až tři prvky pod sebou:
            \begin{itemize}
                \item \textbf{Navigace strukturou} - Jedná se o stejný výčet možností jako v horním menu. Jelikož její obsah závisí na \uv{zapojených} modulech, nebude dále rozebírána.
                \item \textbf{Filtr kategorií} - Poskytuje možnost přepínání filtrace obsahu pro aktuálně zobrazovaný obsah. Jelikož i obsah této nabídky závisí na přidaných skupinách, nebude dále popisována.
                \item \textbf{Podmenu možností} - Nabízí akce dostupné v rámci zobrazovaného obsahu. Změny na jednotlivých záznamech jsou spřaženy s daty, tudíž se neobjevují zde, nýbrž přímo ve výpisu.
            \end{itemize}
            
            \paragraph*{Drobečková navigace}

            Poskytuje navigační nabídku pro snadnější orientaci ve struktuře aplikace. Dále také umožňuje přecházet na rozcestníky, které jsou popsány níže.

            \paragraph*{Notifikace}
            Na tomto místě, přímo pod \textit{drobečkovou navigací} jsou zobrazovány notifikační \uv{zprávičky} o průběhu požadovaných operací, či požadavků na uživatele.

            \paragraph*{Obsah}

            Zde je místo pro prezentaci požadovaných dat a operací s nimi.

        \subsection{Zápatí}
            Místo, pro zobrazení prakticky neměnných dat. V tomto případě je zde umístěna dynamická komponenta s přehledem partnerů klubu. U každého partnera je uvedeno logo, název a odkaz vedoucí na jeho webové stránky. 
            Pod spodním okrajem komponenty s partnery je na pravé straně místo pro užitečné odkazy v rámci aplikace, zahrnující i odkaz na rss kanál novinek (článků).

            V předchozím textu \ref{def-sekce-aplikace} bylo zmíněno, že aplikace se skládá ze 4 sekcí. Pro zpřehlednění orientace ve struktuře nese každá sekce svoji barvu. Uživatelské sekce je označena barvou \textbf{oranžovou}, veřejná sekce \textbf{modrou}, klubová \textbf{červenou} a sekce administrační je \textbf{fialová}. Shodný styl mají i nabídky v horním menu.

        Následující text popisuje jak, a do jakých sekcí zasahují jednotlivé moduly. Je vhodné poznamenat, že samotný modul může ovlivnit pouze obsah výše popsaného těla stránky. Tudíž okolí již nebude nadále dokumentováno. Drobečková navigace je taktéž přítomna na každé stránce (mimo domovskou), tudíž není třeba se o ní nadále zmiňovat.

        \section{Systémový modul}
        TODO Jak ma vypadat struktura tohoto?\\
        TODO Mám sem dávat screen každé sekce? (předpokládám, že ne)\\
        TODO mam zde popisovat i jednotlive formulare?

            \subsection*{Veřejná sekce}
            Veřejná část systémového modulu se skládá z:
            \begin{itemize}
                \item Domovské stránky
                \item Výsledku vyhledávání
                \item Rozcestníků
                \item Chybových výpisů
            \end{itemize}

            \paragraph*{Domovská stránka}
            Je vstupním bodem do aplikace. Skládá se z hlavního motivu a dvousloupcového obsahu. 

            \textit{Hlavním motivem} je přehled nejnovějších zvýrazněných článků napříč kategoriemi.

            \textit{Dvousloupcový obsah} se dále dělí na přehled starších a nezvýrazněných článků a pravého sloupce pro umístění reklamy, novinek, kontaktů a dalších libovolných doplňujících informací.

            V dolní části tohoto obsahu, pod výpisem článků je dále vykreslen plugin poskytující informace o oblíbenosti profilu klubu na Facebooku.

            \paragraph*{Výsledky vyhledávání}
            Vyhledáním slova prostřednictvím formuláře ze záhlaví dojde k vykreslení výsledků vyhledávání právě zde. V horní části přímo pod veřejným menu se mimo jiné nachází \textit{drobečková navigace}. 

            \paragraph*{Rozcestníky}
            V rámci aplikace je možné navštívit celkem dva typy rozcestníků, které se liší jen obsahem a stavem \textit{drobečkové navigace}:
            \begin{itemize}
                \item Rozcestník aplikace - Obsahuje popis struktury všech sekcí.
                \item Rozcestník sekce - Obsahuje popis pouze odpovídající sekce.
            \end{itemize}

            \paragraph*{Chybové výpisy}
            Slouží k informování uživatele, že server odeslanému požadavku \uv{nerozuměl} nebo nebyl schopen nalézt odpověď. Typicky jsou zde zobrazovány chybové odpovědi na HTTP požadavek s kódy 403, 404, 405, 410 a podobné.

            \subsection*{Administrační sekce}

            Úvodní stránka této sekce obsahuje levou navigaci se strukturou administrační sekce a nabídkou dostupných akcí v tomto složení:

            \begin{itemize}
                \item \textit{Přidat sportovní skupinu}
                \item \textit{Přidat typ sportu}
                \item \textit{Přidat statickou stránku}
                \item \textit{Zpět}
            \end{itemize}

            \paragraph*{Přidat sportovní skupinu} 
            Tento odkaz vede na stránku s formulářem pro vytvoření nové sportovní skupiny.

            \paragraph*{Přidat typ sportu}
            Odkaz vede na stránku s formulářem pro přidání nového typu sportu.

            \paragraph*{Přidat statickou stránku}
            Pomocí tohoto odkazu se lze dostat na stránku s formulářem pro přidání nové statické stránky.

            \paragraph*{Zpět}
            Odkaz \textit{Zpět} nás zavede na rozcestník administrační sekce.

            Dále tato úvodní stránka obsahuje přehled všech existujících entit v rámci daného modulu, uspořádaných do přehledných tabulek s dostupnými operacemi u každého záznamu.

            \begin{itemize}
                \item Tabulka \uv{Sportovní skupiny}
                \item Tabulka \uv{Typy sportu}
                \item Tabulka \uv{Statické stránky}
            \end{itemize}

            %\paragraph*{Tabulka Sportovní skupiny}
            U každého záznamu je dostupná operace mazání záznamu a jeho úpravy. Dále tabulky umožňují export dat ve formátu \verb|.csv|.

            - TODO musím zde popisovat každý sloupec nebo stačí jen operace???, ty tabulky umí stejné věci, takže jsem jejich definici vynechal..

            %\paragraph*{Tabulka Typy sportu}

            %\paragraph*{Tabulka Statické stránky}

            %Uprava skupiny
            %Uprava typu
            %Uprava stranky

        \section{Bezpečnostní modul}

            \subsection*{Veřejná sekce}
            V této sekci bezpečnostní modul zobrazuje kontaktní informace o osobách v daných pozicích. V levé části pro menu se nachází filtrační nabídka skupin, v pravé části pro obsah pak samotné kontaktní informace jednotlivých pozic.

            Zobrazení kontaktních informací musí mít daná pozice povoleno. 
            Mimo zobrazovaných kontaktů skupiny je zde permanentně v rámci každé skupiny kontakt na klub spolu s jeho kontaktní osobou.

            Dalším obsahem zobrazovaným v této sekci je přihlašovací stránka, která obsahuje pouze přihlašovací formulář s poli pro zadání přihlašovacího e-mailu a hesla. Dostat se na ni můžeme skrze výše zmíněný\ref{login-link} odkaz v záhlaví stránky.
            
            \subsection*{Administrační sekce}
            Na úvodní stránce této sekce nalezneme opět, v prostoru pro levou navigaci, nabídku možností:
            \begin{itemize}
                \item \textit{Přidat roli}
                \item \textit{Přidat pravidlo}
                \item \textit{Přidat pozici}
                \item \textit{Zpět}
            \end{itemize}

            \paragraph*{Přidat roli} 
            Tento odkaz vede na stránku s formulářem pro přidání role.

            \paragraph*{Přidat pravidlo} 
            Odkaz \textit{Přidat pravidlo} vede na stránku pro přidání autorizačního pravidla vybrané role pro akce jednotlivých sekcí.

            \paragraph*{Přidat pozici} 
            Odkaz vede na stránku sloužící pro přidání pozice daného uzivatele v rámci skupiny.

            \paragraph*{Zpět} 
            Tento odkaz vede opět na rozcestník administrační sekce.
            
            \paragraph*{}
            V části pro obsah jsou umístěny následující tabulky:
            \begin{itemize}
                \item Tabulka \uv{Uživatelské role}
                \item Tabulka \uv{Pravidla}
                \item Tabulka \uv{Pozice uživatelů ve skupinách}
            \end{itemize}

            V každém řádku je možnost mazání a úpravy odpovídajícího záznamu. Dále tabulky umožňují export dat ve formátu \verb|.csv|.

            %Upravu role
            %Upravu pravidla
            %Upravu pozice


        \section{Modul aktualit} 

            \subsection*{Veřejná sekce}
            Veřejná část modulu aktualit obsahuje:
            \begin{itemize}
                \item Přehled článků
                \item Náhled článku
                \item Rss kanál novinek
            \end{itemize}

            \paragraph*{Přehled článků}
             Tento přehled slouží k zobrazení všech zveřejněných článků v kategorii z filtrační nabídky skupin. Ta se nachází v části pro levou navigaci.

             Na tuto stránku je možné se dostat skrze odkaz \textit{Články} ve veřejném menu nebo z drobečkové navigace \textit{náhledu článku}.

            \paragraph*{Náhled článku}\label{doc-nahled-clanku}
            Slouží k zobrazení obsahu článku. Obsahuje nadpis, datum vytvoření a jméno autora. Následuje přiložený obrázek a samotný obsah. V případě, že má článek povoleny komentáře, je dole pod textem formulář pro jejich přidávání následované jejich výpisem. 

            V levé části pro navigaci je, z důvodu urychlení přechodu mezi kategoriemi, umístěna filtrační nabídka skupin. Její dílčí odkazy vedou zpět do \textit{přehledu článků} pro zvolenou skupinu.

            \paragraph*{Rss kanál novinek}
            Slouží pro pravidelný odběr novinek. Odkaz na tuto službu je vždy pod spodním okrajem přehledu partnerů.

            \subsection*{Administrační sekce}
            I v tomto modulu má administrační sekce obdobnou strukturu. Levá navigace obsahuje strukturu administrační sekce. Níže je opět nabídka možností:
            \begin{itemize}
                \item \textit{Přidat článek}
                \item \textit{Zpět}
            \end{itemize}
            
            \paragraph*{Přidat článek} 
            Tento odkaz vede na stránku s formulářem sloužícím pro přidání nového článku.

            \paragraph*{Zpět} 
            Odkaz \textit{Zpět} vede opět na rozcestník sekce.

            \paragraph*{}
            V části pro obsah se nachází tabulka s výpisem všech existujících článků. Každý záznam má k dispozici operace mazání, úpravy a přechodu na \textit{náhled článku}\ref{doc-nahled-clanku}.

        \section{Modul plateb}

            \subsection*{Uživatelská sekce}
            
            prehled
            akce s platbama
            
            Detail platby

            \subsection*{Administrační sekce}

            menu
            moznosti
            prehled

            Zadani platby

            default

            Uprava platby



        \section{Modul sezón}



            \subsection*{Administrační sekce}



            Pridani sezony

            Pridani prihlasky

            Pridani povinnosti

            default

            Uprava sezony

            Uprava prihlasky

            Uprava povinnosti



        \section{Komunikační modul}



            \subsection*{Uživatelská sekce}

            Vytvorit zpravu

            Prehled prijatych

            Prehled odeslanych

            Prehled smazanych

            Odpovedet na zpravu



            \subsection*{Klubová sekce}

            Pridani vlakno fora

            default Zobrazit prehled for

            Zobrazit forum

            Zobrazit vlakno



            \subsection*{Administrační sekce}

            Pridat forum

            Pridat vlakno fora

            default

            Upravit forum

            Upravit vlakno fora



        \section{Modul partnerů}



            \subsection*{Administrační sekce}

            Pridani partnera

            Prehled partneru

            Uprava partnera



        \section{Modul nástěnek}



            \subsection*{Klubová sekce}

            Zobrazit nastenku - prehled

            Zobrazit detail prispevku



            \subsection*{Administrační sekce}

            Pridani prispevku

            Prehled

            Uprava prispevku



        \section{Modul událostí}



            \subsection*{Uživatelská sekce}

            Prehled podilenych udalosti



            \subsection*{Klubová sekce}

            Prehled udalosti

            Detail udalosti



            \subsection*{Administrační sekce}

            Pridani udalosti

            Prehled

            Uprava udalosti

            Uprava ucasti



        \section{Motivační modul}



            

            \subsection*{Uživatelská sekce}

            Prehled motivaci



            \subsection*{Administrační sekce}

            Pridat motivaci

            Pridat cenik

            Prehled

            Uprava motivace

            Uprave ceniku



        \section{Modul uživatelů}



            \subsection*{Veřejná sekce}

            Zobrazit profil

            Zobrazeni soupisek



            \subsection*{Uživatelská sekce}

            Uprava dat

            Uprava hesla

            Uprava profilu





            \subsection*{Administrační sekce}

            Prehled

            Pridat uzivatele

            Upravit uzivatele

            Upravit profil





































%%%%%%%%%%%%%%%%%%%% PROGRAMATOR %%%%%%%%%%%%%%%%%%%%%%%
\chapter{Programátorská dokumentace}

\section{Provedené změny na oficiálních implementacích}

\subsection{Nette\textbackslash Forms\textbackslash Container}
    \begin{itemize}
    \item Byla upravena metoda \verb|setValues|, aby využívala metodu \verb|mineData|
    \item Byla přidána metoda \verb|mineData(array $values, $name)|, která pracuje s rozhranním \verb|IIdentifiable|
    \end{itemize}

\section{Konfigurace}
- popis a rozbor implementační struktury aplikace (tohle by mělo být spíš v poslední části analýzy)
- struktura modulu






























%%%%%%%%%%%%%%%%%%%%%%%%%% ZAVER %%%%%%%%%%%%%%%%%%%%%%
\chapter{Závěr}
Klasicky zaver, k cemu jsem dospel. Navrhy pro rozsireni.
Co se s aplikace bude dit dal.






























%%%%%%%%%%%%%%%%%%%%%%%%%%% PRILOHY %%%%%%%%%%%%%%%%%%%%%%%
\chapter{Přílohy}
- určitě dotazník
- vyhodnoceni dotazniku
- odpovedi na dotaznik
- obrázky
- tabulky
- s nejvyšší pravděpodobností programátorská dokumentace vygenerovaná apigenem

%\bibliographystyle{plain}  % bibliografický styl 
%\bibliography{diploma-thesis} % soubor s citovanými
                           % položkami bibliografie 

%\renewcommand\bibname{Literatura}  
\begin{thebibliography}{99}

\bibitem{progit} CHACON, SCOTT: Apress; 1 edition (August 26, 2009), 288 s. ISBN 1430218339. Pro Git.

\bibitem{zend-hydrator-api} Zend hydrator api\\
\url{http://apigen.juzna.cz/doc/zendframework/zf2/class-Zend.Stdlib.Hydrator.ClassMethods.html}

\bibitem{php-net} Php dokumentace\\
\url{http://www.php.net/docs.php}

\bibitem{analyza-sis} TRIPES, STANISLAV: Vysoká škola ekonomická v Praze, Diplomová práce, Informační systémy pro sportovní kluby\\
\url{http://www.vse.cz/vskp/28467_informacni_systemy_pro_sportovni_kluby}

\bibitem{racek} RÁČEK, J.: Strukturovaná analýza systémů. Masarykova univerzita, 2006, ISBN\\
80-210-4190-0.

\bibitem{grido} Dokumentace grido rozšíření\\
\url{http://o5.github.io/grido-sandbox/documentation.cs.html#client-side-features}

\bibitem{composer} Oficiální web pro Composer\\
\url{https://getcomposer.org/}

\bibitem{html5-all} Server zabývající se Html5 problematikou\\
\url{http://www.html5.cz/}

\bibitem{html5-new} Nové vlastnosti Html5 - server Programujte.cz\\
\url{http://programujte.com/clanek/2010082200-html5-nove-vlastnosti/}

\bibitem{doctrine} Oficiální web projektu Doctrine\\
\url{http://www.doctrine-project.org/}

\bibitem{symfony} Oficiální web projektu Symfony
\url{http://symfony.com}

\bibitem{kdyby} Oficiální web CMS Kdyby
\url{https://www.kdyby.org/}

\bibitem{nette} Oficiální web Nette frameworku
\url{http://www.nette.org}

\bibitem{apache} Oficiální web projektu Apache HTTP server
\url{http://httpd.apache.org/}

\bibitem{dibi} Oficiální web projektu Dibi
\url{http://www.dibiphp.com}

\bibitem{php-lin} Článek na serveru Linuxsoft 
\url{http://www.linuxsoft.cz/article.php?id_article=171}

\end{thebibliography}






























%%%%%%%%%%%%%%%%%%%%%%%%%%%%%%% DODATEK %%%%%%%%%%%%%%%%%%%%%%%%%%%%%%%
\newpage
\appendix
\chapter{Dodatek}
Informace o pouzitych doplncich s uvedenim autorstvi pro pripad, ze jejich licence nedovoluje volne sireni nebo pozaduje uvedeni autorstvi. Pripadne upravy prevzatych del jsou zde take uvedeny.
V aplikaci jsou použity a dále upraveny komponenty TODO 1, TODO 2, TODO 3, všechny jsou volně dostupné na serveru \verb|http://gist.github.com| nebo na \verb|http://nette.org|.

Grafické prvky použité v~aplikaci jsou převzaté ze serveru \verb|TODO 4| a~dále upraveny v~aplikaci GIMP Image Editor verze \verb|TODO 5|.

\newpage
\chapter{Obsah přiloženého CD} 

\begin{description}

\item[\texttt{bin/}] \hfill \\
Kompletní a minimalizovaná adresářová struktura webové aplikace (v ZIP archivu) pro zkopírování na webový server a bezproblémový provoz.

\item[\texttt{doc/}] \hfill \\
Dokumentace práce ve formátu PDF, vytvořená dle závazného stylu KI PřF pro diplomové práce. Dále adresář obsahuje všechny zdrojové soubory nutné k vytvoření dokumentace.

\item[\texttt{src/}] \hfill \\
Kompletní zdrojové texty webové aplikace se všemi potřebnými (převzatými) zdrojovými texty,
knihovnami a~dalšími soubory adresářové struktury pro zkopírování na webový server (v ZIP archivu).

\item[\texttt{readme.txt}] \hfill \\
Instrukce pro nasazení webové aplikace na webový server, včetně požadavků pro její provoz, a
webová adresa, na které je aplikace nasazena pro testovací účely a~pro účel obhajoby práce.

\end{description}
U veškerých odjinud převzatých materiálů obsažených na CD jejich zahrnutí dovolují podmínky pro jejich šíření. Pro materiály, u~kterých toto není splněno, je uveden jejich zdroj v~textu dokumentace práce.

\end{document}




















%%%%%%%%%%%%%%%%%%%% NAVRH NA ZLEPSENI %%%%%%%%%%%%%%%%%%%%%%%%

%\paragraph*{}
%Ajax v Nette, když bude málo textu.

% zminit jak funguji ty rozsireni a vubec kompilace v nette, pac je to na tom postaveny, zj..


%%%% POZNAMKY %%%%

%Php traits
%http://php.net/manual/en/language.oop5.traits.php

%Grido example
%https://github.com/o5/grido/blob/master/tests/Grido/DataSources/Doctrine.phpt#L69-L73

%% Konverze tiff do pdf
%%% tiff2pdf modul_clenske_zakladny.tif > clenove.pdf


%%%% README %%%%

%- potreba mit spravne nastavena prava pro adresar temp pro zapis do cache/ 
%- uvest potrebne moduly pro beh aplikace.
%php5 module mem_cache musi byt nainstalovano

%%%% KONEC? %%%%
%%% Jestli si opravdu myslis, ze uz mas vsechno tak udeleje nasledujici:
%% Jeste si to jednou projed, mrkni se i na klicovy slova
%% zkus najit TODO slova
%% projdi si navrhy na zlepseni, jestli nejsou prazdny
%% projdi si seznam, jestli dava smysl
%% projdi si jednotlive odstavce jestli se to dobre cte a navazuje na sebe
%% Mas v aplikaci komentovany kod? Myslim kazdou metodu i privatni...


%Vypadá to dobře, se strukturou souhlasím, možná tu programátorskou dokumentaci pak jenom do přílohy, ale zase modifikace frameworku dejte klidně do textu. Stejně tak jak tam máte dotaz na tu instalaci - to by také patřilo lépe do přílohy (ideálně pak i do README souboru k balíku).

%K těm technologiím web serveru, databáze a nette bych dal krátké (odstavec nebo dva) vysvětlení, co to je a k čemu je to dobré (ale fakt stručně). A pak případně můžete rozebrat potřebné moduly, které používáte nebo nějaké speciální nastavení (běžnou konfiguraci spíše do instalační příručky) - sem lze například napsat, že u mysql se používají transakce nebo pro php rewrite pro clean url nebo něco v tom duchu. A určitě bych zmínil i ORM mapování a ten security modul, případně taky autentizaci (teď si nevzpomínám, jestli používáte něco standardního, pak bych to dal sem, nebo vlastního, pak bych tomu věnoval sekci v implementaci).





